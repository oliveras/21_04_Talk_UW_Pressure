\section{Alternative Formulations}

% ==============================================================================================

\begin{frame}[t]\frametitle{Equations of Motion - %
    \only<1-2>{Velocity Potential Formulation}%
    \only<3>{ZCS Formulation}%
    \only<4->{AFM Formulation}}
    \begin{overlayarea}{\textwidth}{.56\textheight}
        \only<1-2>{%
            We begin by considering an inviscid, irrotational, fluid with a one-dimensional free-surface on the whole-line:
            \vspace*{-.12in}
            \begin{align*}
                &\phi_{xx} + \phi_{zz} =0,  & &(x,z) \in \mathscr{D}, \\ 
                &\phi_t + \frac{1}{2}\vert\nabla\phi\vert^2 + \frac{\rho g z - p(x,z,t)}{\rho} = 0,& &(x,z) \in \mathscr{D}, \\ 
                &\phi_z =0, & &z = -h, \\
                &\eta_t  =\phi_z-\eta_x\phi_x,  &&z = \eta(x,t),\\ 
                &p =0, && z = \eta(x,t),
            \end{align*}
            \vspace*{-.08in}
            where \(\phi(x,z,t)\) represents the velocity potential of the fluid, \(\eta(x,t)\) represents the surface elevation.
        }
        \only<3>{
            \rcite{Zakharov, and Craig \& Sulem (ZCS)}: \hfill\emA{\(\boxed{q(x,t) = \phi(x, \eta(x,t),t)}\)} \\
            \vspace*{.1in}

            \emA{{\bf{Dynamic Boundary Condition}}}\\
            \hspace*{1.5in}\(\displaystyle q_t + \frac{1}{2} q_x^{2}  + g\eta  - \frac{1}{2}\frac{\left( \eta_t  +  \eta_{x}q_x\right)^{2}}{1+\eta_{x}^{2}} = 0, \)\\
            
            \vspace*{.1in}

            \emA{{\bf{Kinematic Boundary Condition}}}\\
            \hspace*{1.5in}\(\displaystyle  \eta_t = \mathcal{G}( \eta)q\),  \hfill\emA{\(\boxed{\mathcal{G}( \eta)q = \normD{\phi}, ~~z =  \eta}\)}\\
            }
        \only<4->{
            \rcite{Ablowitz, Fokas, \& Musslimani (AFM)}: \hfill\emA{\(\boxed{q(x,t) = \phi(x, \eta(x,t),t)}\)}\\
            \vspace*{.1in}

            \emA{{\bf{Dynamic Boundary Condition/Local Equation}}}\\
            \hspace*{1.5in}\(\displaystyle q_t + \frac{1}{2} q_x^{2}  + g\eta  - \frac{1}{2}\frac{\left( \eta_t  +  \eta_{x}q_x\right)^{2}}{1+\eta_{x}^{2}} = 0, \)\\
            
            \vspace*{.1in}

            \emA{{\bf{Kinematic Boundary Condition/Nonlocal Equation}}}\\
                \[\wlint{e^{-ikx}\left( \eta_t\cosh(k(\eta+h))+iq_x\sinh(k(\eta+h)) \right)}=0\]
            }
    \end{overlayarea}
    \vspace*{.1in}

    \begin{overlayarea}{\textwidth}{.4\textheight}
        \drawFigure{
            \plotAir
            \expWave
            \plotBottom
            \plotPressureSensor
            \plotAxes{\(x\)}{\(z\)}
            \uput[r](-.5,0){\(\mathscr{D}\)}
            \uput[l](-7,-1){\(z = -h\)}
            \uput[r](1,3){\(z = \eta(x,t)\)}
            \pause%
            \uput[u](0,4.25){\emB{\(\boxed{\textsf{\bf Transform to Boundary Variables}}\)}}
        }
    \end{overlayarea}
    
\end{frame}





% % ==============================================================================================
% \begin{frame}[t]\frametitle{Equations of Motion - %
%                                                 \only<1>{Velocity Potential Formulation}%
%                                                 \only<2>{ZCS Formulation}%
%                                                 \only<3->{AFM Formulation}}
%     \begin{overlayarea}{\textwidth}{.6\textheight}
%         \only<1>{
%            What are the various ways to transform these equations into surface variables?
%             \begin{align*}
%                 &\phi_{xx} + \phi_{zz} =0,  & &(x,z) \in \mathscr{D}, \\ 
%                 &\phi_z =0, & &z = -h, \\
%                 & \eta_t  =\phi_z- \eta_x\phi_x,  &&z =  \eta(x,t),\\ 
%                 &\phi_t + \frac{1}{2} \left(\phi_x^2 + \phi_z^2\right) +  g\eta =0, && z =  \eta(x,t),
%             \end{align*}
%             where \(\phi(x,z,t)\) represents the velocity potential of the fluid, \(\eta(x,t)\) represents the surface elevation.
%         }
%         \only<2>{
%             \rcite{Zakharov, and Craig \& Sulem (ZCS)}: \hfill\emA{\(\boxed{q(x,t) = \phi(x, \eta(x,t),t)}\)} \\
%             \vspace*{.1in}

%             \emA{{\bf{Dynamic Boundary Condition}}}\\
%             \hspace*{1.5in}\(\displaystyle q_t + \frac{1}{2} q_x^{2}  + g\eta  - \frac{1}{2}\frac{\left( \eta_t  +  \eta_{x}q_x\right)^{2}}{1+\eta_{x}^{2}} = 0, \)\\
            
%             \vspace*{.1in}

%             \emA{{\bf{Kinematic Boundary Condition}}}\\
%             \hspace*{1.5in}\(\displaystyle  \eta_t = \mathcal{G}( \eta)q\),  \hfill\emA{\(\boxed{\mathcal{G}( \eta)q = \normD{\phi}, ~~z =  \eta}\)}\\
%             }
%         \only<3->{
%             \rcite{Ablowitz, Fokas, \& Musslimani (AFM)}: \hfill\emA{\(\boxed{q(x,t) = \phi(x, \eta(x,t),t)}\)}\\
%             \vspace*{.1in}

%             \emA{{\bf{Dynamic Boundary Condition/Local Equation}}}\\
%             \hspace*{1.5in}\(\displaystyle q_t + \frac{1}{2} q_x^{2}  + g\eta  - \frac{1}{2}\frac{\left( \eta_t  +  \eta_{x}q_x\right)^{2}}{1+\eta_{x}^{2}} = 0, \)\\
            
%             \vspace*{.1in}

%             \emA{{\bf{Kinematic Boundary Condition/Nonlocal Equation}}}\\
%                 \[\wlint{e^{-ikx}\left( \eta_t\cosh(k(\eta+h))+iq_x\sinh(k(\eta+h)) \right)}=0\]
%             }
%     \end{overlayarea}
%     \begin{overlayarea}{\textwidth}{.4\textheight}
%         \only<1-3>{
%             \drawFigure{
%                 \plotAir
%                 \expWave
%                 \plotBottom
%                 \plotPressureSensor
%                 \plotAxes{\(x\)}{\(z\)}
%                 \uput[r](-.5,0){\(\mathscr{D}\)}
%                 \uput[l](-7,-1){\(z = -h\)}
%                 \uput[r](1,3){\(z = \eta(x,t)\)}
%             }
%         }
%         \only<4->{
%             \vspace*{-.375in}
%             \emA{{\bf{A Third Equation}}}\hfill\emA{\(\boxed{Q(x,t) = \phi(x,-h,t)}\)}\\
%             \hfill \(\displaystyle \wlint{e^{-ikx}\left( \eta_t\sinh(k(\eta+h))-iq_x\cosh(k(\eta+h)) \right)}=k\hat{Q}(k,t)\)\hfill ~\\
%         }
        
%     \end{overlayarea}
% \end{frame}
% ==============================================================================================
\begin{frame}[t]
    \frametitle{Mapping Pressure via AFM}
    From the previous slide:\\~\\
    \emA{{\bf{A Third Equation}}}\hfill\emA{\(\boxed{Q(x,t) = \phi(x,-h,t)}\)}\\
    \hfill \(\displaystyle \wlint{e^{-ikx}\left( \eta_t\sinh(k(\eta+h))-iq_x\cosh(k(\eta+h)) \right)}=k\hat{Q}(k,t)\)\hfill ~\\

    \pause%
    {
        \vspace*{.1in}
       \begin{center}\(\hat{Q}(k,t) \to Q(x,t) \qquad \emC{\Rightarrow}\qquad \displaystyle Q_t + \frac{1}{2}Q_x^2 + \frac{p(x,-h,t)-\rho g h}{\rho} = 0\) \end{center}
    }
    \pause
    For traveline waves, you can find the following relationships:
    \[p(\xi,-h) - \rho g h = \rho \iint_{\mathbb{R}\times\mathbb{R}}e^{ik(\xi - \xi')}\sqrt{(c^2 - 2 g \eta)(1 + \eta_x^2)}\cosh(k(\eta + h))\,d\xi'\, dk\]
    or via a separate derivation
    \[\int_{-\infty}^{\infty} e^{-i k \xi}\sqrt{c^2 - 2(p - \rho g h)}\cosh(k(\eta + h))\,dk =\frac{\sqrt{c^2 - 2g\eta}}{1 + \eta_\xi^2}\]
\end{frame}
% ==============================================================================================

